% Create a Table of Contents in Beamer
\documentclass[10pt,t]{beamer}
% Theme choice:
\usetheme{Singapore}
\useoutertheme{sidebar}

\usecolortheme{seahorse}
\setbeamercolor{titlelike}{bg=white}
\setbeamercolor{frametitle}{bg=white}
%\setbeamertemplate{frametitle}[default][left]
\setbeamertemplate{navigation symbols}{}
\setbeamertemplate{footline}{\begin{flushright}\small \insertframenumber\end{flushright}}

\usepackage{graphicx}
\usepackage{amsmath}
\usepackage{amsfonts}
\usepackage{amssymb}
\usepackage{amsthm}
\usepackage{ulem}
\usepackage{listings}
\usepackage{hyperref}

% Title page details: 
\title{Language and variable definitions in public health and this course} 
\subtitle{Especially around substance use, sex, gender, race, and ethnicity}
\author{Nina Galanter}
\date{\today}


\begin{document}
	% Title page frame
	\begin{frame}
	\titlepage 
\end{frame}

\begin{frame}{Person-first language\footnote{Carey Farquhar EPI 530 Fall 2021}}
	\vspace{-5 mm}
	
In this course, as is widespread in public health, we will use person-first language.
\medskip
\begin{itemize}
\item Focuses on the person instead of their health
status, condition, or behavior
\medskip
\item Describes what a person has or does, not who a
person is
\medskip
\item Goals
\medskip
\begin{itemize}
\item Avoid reducing a person to a specific characteristic
\medskip
\item Avoid perpetuating stigma and/or ignorance
\end{itemize}
\medskip
\item For example: people who inject drugs, subjects with diabetes, person living with HIV, people who smoke
\end{itemize}
\end{frame}

\begin{frame}{Person-centered language\footnote{Carey Farquhar EPI 530 Fall 2021}\footnote{\url{https://radicalcopyeditor.com/2017/07/03/person-centered-language/}}}
	\vspace{-5 mm}
	
	While many prefer person-first language, not all
	persons or groups do
	\medskip
	\begin{itemize}
	\item Some feel certain characteristics are inseparable from identity
	\medskip
	\item Some argue that separating the person from
	their condition or experience can make it feel
	like something to be ashamed of
	\medskip
	\item Goal is to treat people as the
	“experts on themselves” and use the language
	they prefer – person-first or otherwise
	\end{itemize}
\end{frame}

\begin{frame}{Language around drug use\footnote{Biostatstics EDI Comittee: \url{https://docs.google.com/document/d/1tem4zDGRCdl5jcImrv_UoVZ5D-IUsm5mUwNrSurQd08/edit}}}
	\vspace{-5 mm}
	
	\begin{itemize}
	\item Drug use is often very stigmatized
	\medskip
	\item In the US substance use disorders are often blamed on people who live with these conditions
	\medskip
	\item Unfair and hides the role of government policies and companies (such as \textcolor{violet}{\href{https://www.washingtonpost.com/national/hammer-on-the-abusers-mass-attorney-general-alleges-purdue-pharma-tried-to-shift-blame-for-opioid-addiction/2019/01/15/4af25c4c-190c-11e9-88fe-f9f77a3bcb6c_story.html}{pharmaceutical companies that produce opioids}})
	\medskip
	\item Unhelpful and harmful from a public health perspective 
	\medskip
	\item To prevent this we can avoid "drug abuse" and use person-first language
	\medskip
	\item We can label variables as behaviors (e.g. smoking, opioid use) rather than identities (e.g. smokers, opioid users) 
\end{itemize}
\end{frame}

\begin{frame}{Sex and Gender\footnote{Biostatstics EDI Comittee: \url{https://docs.google.com/document/d/1tem4zDGRCdl5jcImrv_UoVZ5D-IUsm5mUwNrSurQd08/edit}}}
	\vspace{-7 mm}
	
	\begin{itemize}
	\item sex is a biological variable related to chromosomes and/or primary/secondary sex characteristics
	\smallskip
	\item gender is non-biological and related to social presentation and self-perception
	\smallskip
	\item In biomedical studies it is much more common to consider sex than gender, though sex may sometimes be mislabeled as gender
	\smallskip
	\item Depending on how data is collected, it may be unclear whether a variable represents sex, gender, or a complicated combination
	\smallskip
	\item Sex is not a binary variable but is usually binarized in datasets
	\smallskip
	\begin{itemize}
		\item in statistics, we may not be able to work with observations which fall into a small subgroup; collapsing categories can prevent this
		\smallskip
		\item however, binarizing sex may marginalize people who are intersex
		\smallskip
		\item there's not an easy answer
	\end{itemize}
	\end{itemize}
\end{frame}

\begin{frame}{Race and Ethnicity\footnote{Biostatstics EDI Comittee:: \url{https://docs.google.com/document/d/1tem4zDGRCdl5jcImrv_UoVZ5D-IUsm5mUwNrSurQd08/edit}}}
	\vspace{-7 mm}
	
	\begin{itemize}
		\item Race and ethnicity are socially and politically defined identities which impact our health
		\smallskip
		\item However, the collection/use of data on race and ethnicity often:
		\smallskip
		\begin{itemize}
			\item excludes people who are multiracial
			\medskip
			\item excludes people who do not fit into a small number of categories
			\medskip
			\item can group together people with very different identities in terms of race and ethnicity
			\smallskip
		\end{itemize}
	\item Again there is a tension as we are limited by sample size in
\smallskip
	\begin{itemize}
		\item how many variables we can include in an analysis
		\medskip
		\item whether we can analyze data on certain subgroups
	\end{itemize}
\smallskip
\item One solution when designing a study is to over-sample people of certain identities
	\end{itemize}
\end{frame}

\end{document}