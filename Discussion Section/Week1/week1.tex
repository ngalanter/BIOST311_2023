% Options for packages loaded elsewhere
\PassOptionsToPackage{unicode}{hyperref}
\PassOptionsToPackage{hyphens}{url}
%
\documentclass[
]{article}
\usepackage{amsmath,amssymb}
\usepackage{lmodern}
\usepackage{iftex}
\ifPDFTeX
  \usepackage[T1]{fontenc}
  \usepackage[utf8]{inputenc}
  \usepackage{textcomp} % provide euro and other symbols
\else % if luatex or xetex
  \usepackage{unicode-math}
  \defaultfontfeatures{Scale=MatchLowercase}
  \defaultfontfeatures[\rmfamily]{Ligatures=TeX,Scale=1}
\fi
% Use upquote if available, for straight quotes in verbatim environments
\IfFileExists{upquote.sty}{\usepackage{upquote}}{}
\IfFileExists{microtype.sty}{% use microtype if available
  \usepackage[]{microtype}
  \UseMicrotypeSet[protrusion]{basicmath} % disable protrusion for tt fonts
}{}
\makeatletter
\@ifundefined{KOMAClassName}{% if non-KOMA class
  \IfFileExists{parskip.sty}{%
    \usepackage{parskip}
  }{% else
    \setlength{\parindent}{0pt}
    \setlength{\parskip}{6pt plus 2pt minus 1pt}}
}{% if KOMA class
  \KOMAoptions{parskip=half}}
\makeatother
\usepackage{xcolor}
\IfFileExists{xurl.sty}{\usepackage{xurl}}{} % add URL line breaks if available
\IfFileExists{bookmark.sty}{\usepackage{bookmark}}{\usepackage{hyperref}}
\hypersetup{
  pdftitle={Introduction to R and RStudio},
  pdfauthor={Nina Galanter},
  hidelinks,
  pdfcreator={LaTeX via pandoc}}
\urlstyle{same} % disable monospaced font for URLs
\usepackage[margin=1in]{geometry}
\usepackage{color}
\usepackage{fancyvrb}
\newcommand{\VerbBar}{|}
\newcommand{\VERB}{\Verb[commandchars=\\\{\}]}
\DefineVerbatimEnvironment{Highlighting}{Verbatim}{commandchars=\\\{\}}
% Add ',fontsize=\small' for more characters per line
\usepackage{framed}
\definecolor{shadecolor}{RGB}{248,248,248}
\newenvironment{Shaded}{\begin{snugshade}}{\end{snugshade}}
\newcommand{\AlertTok}[1]{\textcolor[rgb]{0.94,0.16,0.16}{#1}}
\newcommand{\AnnotationTok}[1]{\textcolor[rgb]{0.56,0.35,0.01}{\textbf{\textit{#1}}}}
\newcommand{\AttributeTok}[1]{\textcolor[rgb]{0.77,0.63,0.00}{#1}}
\newcommand{\BaseNTok}[1]{\textcolor[rgb]{0.00,0.00,0.81}{#1}}
\newcommand{\BuiltInTok}[1]{#1}
\newcommand{\CharTok}[1]{\textcolor[rgb]{0.31,0.60,0.02}{#1}}
\newcommand{\CommentTok}[1]{\textcolor[rgb]{0.56,0.35,0.01}{\textit{#1}}}
\newcommand{\CommentVarTok}[1]{\textcolor[rgb]{0.56,0.35,0.01}{\textbf{\textit{#1}}}}
\newcommand{\ConstantTok}[1]{\textcolor[rgb]{0.00,0.00,0.00}{#1}}
\newcommand{\ControlFlowTok}[1]{\textcolor[rgb]{0.13,0.29,0.53}{\textbf{#1}}}
\newcommand{\DataTypeTok}[1]{\textcolor[rgb]{0.13,0.29,0.53}{#1}}
\newcommand{\DecValTok}[1]{\textcolor[rgb]{0.00,0.00,0.81}{#1}}
\newcommand{\DocumentationTok}[1]{\textcolor[rgb]{0.56,0.35,0.01}{\textbf{\textit{#1}}}}
\newcommand{\ErrorTok}[1]{\textcolor[rgb]{0.64,0.00,0.00}{\textbf{#1}}}
\newcommand{\ExtensionTok}[1]{#1}
\newcommand{\FloatTok}[1]{\textcolor[rgb]{0.00,0.00,0.81}{#1}}
\newcommand{\FunctionTok}[1]{\textcolor[rgb]{0.00,0.00,0.00}{#1}}
\newcommand{\ImportTok}[1]{#1}
\newcommand{\InformationTok}[1]{\textcolor[rgb]{0.56,0.35,0.01}{\textbf{\textit{#1}}}}
\newcommand{\KeywordTok}[1]{\textcolor[rgb]{0.13,0.29,0.53}{\textbf{#1}}}
\newcommand{\NormalTok}[1]{#1}
\newcommand{\OperatorTok}[1]{\textcolor[rgb]{0.81,0.36,0.00}{\textbf{#1}}}
\newcommand{\OtherTok}[1]{\textcolor[rgb]{0.56,0.35,0.01}{#1}}
\newcommand{\PreprocessorTok}[1]{\textcolor[rgb]{0.56,0.35,0.01}{\textit{#1}}}
\newcommand{\RegionMarkerTok}[1]{#1}
\newcommand{\SpecialCharTok}[1]{\textcolor[rgb]{0.00,0.00,0.00}{#1}}
\newcommand{\SpecialStringTok}[1]{\textcolor[rgb]{0.31,0.60,0.02}{#1}}
\newcommand{\StringTok}[1]{\textcolor[rgb]{0.31,0.60,0.02}{#1}}
\newcommand{\VariableTok}[1]{\textcolor[rgb]{0.00,0.00,0.00}{#1}}
\newcommand{\VerbatimStringTok}[1]{\textcolor[rgb]{0.31,0.60,0.02}{#1}}
\newcommand{\WarningTok}[1]{\textcolor[rgb]{0.56,0.35,0.01}{\textbf{\textit{#1}}}}
\usepackage{graphicx}
\makeatletter
\def\maxwidth{\ifdim\Gin@nat@width>\linewidth\linewidth\else\Gin@nat@width\fi}
\def\maxheight{\ifdim\Gin@nat@height>\textheight\textheight\else\Gin@nat@height\fi}
\makeatother
% Scale images if necessary, so that they will not overflow the page
% margins by default, and it is still possible to overwrite the defaults
% using explicit options in \includegraphics[width, height, ...]{}
\setkeys{Gin}{width=\maxwidth,height=\maxheight,keepaspectratio}
% Set default figure placement to htbp
\makeatletter
\def\fps@figure{htbp}
\makeatother
\setlength{\emergencystretch}{3em} % prevent overfull lines
\providecommand{\tightlist}{%
  \setlength{\itemsep}{0pt}\setlength{\parskip}{0pt}}
\setcounter{secnumdepth}{-\maxdimen} % remove section numbering
\usepackage{color}
\usepackage{fvextra}
\DefineVerbatimEnvironment{Highlighting}{Verbatim}{breaklines,commandchars=\\\{\}}
\ifLuaTeX
  \usepackage{selnolig}  % disable illegal ligatures
\fi

\title{Introduction to R and RStudio}
\usepackage{etoolbox}
\makeatletter
\providecommand{\subtitle}[1]{% add subtitle to \maketitle
  \apptocmd{\@title}{\par {\large #1 \par}}{}{}
}
\makeatother
\subtitle{BIOST 311, Discussion Section Week 1}
\author{Nina Galanter}
\date{March 28th, 2023}

\begin{document}
\maketitle

{
\setcounter{tocdepth}{2}
\tableofcontents
}
\hypertarget{r-the-big-picture}{%
\subsection{R: the big picture}\label{r-the-big-picture}}

R is a \textcolor{blue}{free}, \textcolor{orange}{open source} software
package that can be used for data analysis, graphics, and programming.

At its core, R is an interactive, command-driven
\textcolor{blue}{language}: you type a command, R executes the command,
and then returns results.

While it may take some time to develop a mastery of R, it is relatively
easy to get set up with the basics and start analyzing data.

Learning R is worth your time, and has many advantages, including:

\begin{itemize}
\tightlist
\item
  R is free and open source
\item
  there is an active group of contributors to R software
\item
  you can do flexible things with your data
\item
  many people have written packages that make data analysis easier
\end{itemize}

To begin using R, you need to download and install it. You can download
the latest version of R \href{https://www.r-project.org/}{here}.

\hypertarget{rstudio-the-interface-to-r}{%
\subsection{RStudio: the interface to
R}\label{rstudio-the-interface-to-r}}

While R is the \textcolor{blue}{language} you will use to analyze data,
it does not come packaged with the best \textcolor{orange}{interface}.
An interface connects you to the program: for example, your operating
system lets you point and click on icons (and read things easily) rather
than writing in 0's and 1's, the language that the computer understands.

Instead of using the R interface packaged with the download, you will
use RStudio. This is also a \textcolor{blue}{free},
\textcolor{orange}{open source} software package, but is only an
interface to R---this means that even if you type into RStudio, the
commands will be executed in R. Technically, RStudio is called an
integrated development environment (IDE).

RStudio adds many things over the basic R interface, including:

\begin{itemize}
\tightlist
\item
  An improved layout of tools for data analysis
\item
  Support for embedding reproducible research tools into data analysis
\item
  Support for building your own R packages
\item
  Integrated R help and documentation
\item
  De-bugging tools
\end{itemize}

The combination of R and RStudio makes it easy to:

\begin{itemize}
\tightlist
\item
  write R scripts containing all code for an analysis, along with
  comments describing what the code does
\item
  write reports with code embedded (in Rmarkdown, more on this later)
\item
  organize your data analysis workflow (e.g., reading in data, accessing
  help files)
\end{itemize}

As with R, you need to download RStudio before you can get started. You
can download RStudio \href{https://www.r-project.org/}{here}.

When you open RStudio for the first time, the window will display three
\textcolor{blue}{panels}---these have different functions, as you'll now
discover.

\hypertarget{the-console}{%
\subsubsection{The console}\label{the-console}}

The default position of the console is in the lower left-hand pane of
RStudio. This pane is your real window into R: any commands that you run
in R will appear in the console, and any results that are output by
these commands, with the exception of plots, will appear in the console.

To enter a command into the console, move your cursor to the
\texttt{\textgreater{}} and click. When there is a flashing
\texttt{\textbar{}} in the console, you're ready to enter a command.
From now on, you'll call the \texttt{\textgreater{}} the execution line.
Once you've entered a command on the execution line, hit \texttt{Enter}
to run the command.

For example, you could type in the number \texttt{47} and hit enter,
which prints out the number \texttt{47}:

\begin{Shaded}
\begin{Highlighting}[]
\DecValTok{47}
\end{Highlighting}
\end{Shaded}

\begin{verbatim}
## [1] 47
\end{verbatim}

Or, you could use R as a calculator:

\begin{Shaded}
\begin{Highlighting}[]
\DecValTok{32} \SpecialCharTok{+} \DecValTok{15}
\end{Highlighting}
\end{Shaded}

\begin{verbatim}
## [1] 47
\end{verbatim}

\begin{Shaded}
\begin{Highlighting}[]
\DecValTok{52} \SpecialCharTok{{-}} \DecValTok{5}
\end{Highlighting}
\end{Shaded}

\begin{verbatim}
## [1] 47
\end{verbatim}

\begin{Shaded}
\begin{Highlighting}[]
\DecValTok{94}\SpecialCharTok{/}\DecValTok{2}
\end{Highlighting}
\end{Shaded}

\begin{verbatim}
## [1] 47
\end{verbatim}

\begin{Shaded}
\begin{Highlighting}[]
\DecValTok{23}\SpecialCharTok{*}\DecValTok{2} \CommentTok{\# This is a comment! Everything after a \textquotesingle{}\#\textquotesingle{} isn\textquotesingle{}t run by R; instead, it\textquotesingle{}s printed out. Here, you could say that the result isn\textquotesingle{}t 47!}
\end{Highlighting}
\end{Shaded}

\begin{verbatim}
## [1] 46
\end{verbatim}

Each of these lines returns an \textcolor{blue}{object}---in each case,
the object is the result of the computation. In fact, these objects are
called \textcolor{blue}{vectors}. Vectors are the building block of R
objects.

\hypertarget{the-script-editor}{%
\subsubsection{The script editor}\label{the-script-editor}}

It is possible to do all of your work in the console. However, this
isn't advisable, because:

\begin{itemize}
\tightlist
\item
  you may forget the commands you've run, and
\item
  you may forget the objects you've created;
\end{itemize}

both of these (and other reasons) make \textcolor{blue}{reproducibility}
difficult if you only use the console.

Reproducibility is important both for yourself and for others:

\begin{itemize}
\tightlist
\item
  you should always structure your code so that future you can come back
  in 6 months and figure out what you are doing
\item
  you may have to run your code more than once, so why not write it down
  only once?
\item
  someone else should be able to take your code and data and obtain the
  same results as you did
\end{itemize}

The \textcolor{blue}{script editor} is one place to write down and save
your code. By default, it is located in the upper left panel of RStudio.
This pane serves as a text editor; it allows you to edit and save code
and Rmarkdown documents, among other things.

There are two primary ways to write and save R code: RMarkdown files and
R files.

To create a new RMarkdown file, click on
\texttt{File\ \textgreater{}\ New\ File\ \textgreater{}\ R\ Markdown}.
RMarkdown files are saved with the file extension \texttt{.Rmd}. These
files allow you to seamlessly integrate code with text (this document is
written in Rmarkdown!), and also include graphs, plots, tables, etc. You
can then output your RMarkdown file to Word (.docx), PDF (.pdf), or HTML
(.html). You create the output by clicking \texttt{Knit}.

To create an R script file, click on
\texttt{File\ \textgreater{}\ New\ File\ \textgreater{}\ R\ Script}.
These files, saved with the file extension \texttt{.R}, allow you to
save R code and execute it in the console. They are simpler than
\texttt{.Rmd} files, but also much more limited. One advantage of using
RMarkdown over R scripts is that you can write text explaining what your
code does!

To run code from the source editor, use \texttt{ctrl\ +\ Enter} (PC) or
\texttt{cmd\ +\ Enter} (Mac), which allows you to either run a single
line (whichever line your cursor is on); or multiple lines, by
highlighting multiple lines. (You can also use your mouse by clicking
\texttt{Run\ \textgreater{}\ Run\ Selected\ Lines}, but this quickly
becomes tedious.)

Practice running the following line of code from your .Rmd file into the
console:

\begin{Shaded}
\begin{Highlighting}[]
\DecValTok{45} \SpecialCharTok{+} \DecValTok{2}
\end{Highlighting}
\end{Shaded}

\begin{verbatim}
## [1] 47
\end{verbatim}

The places in the .Rmd file that begin and end with three back-ticks are
called \textcolor{blue}{code chunks}. This is where you enter any code
that you want to be executed by R when you knit your .Rmd file. You can
also test your code by running the code from the chunks in the console.
Use \texttt{ctrl\ +\ shift\ +\ Enter} or
\texttt{cmd\ +\ shift\ +\ enter} to run an entire chunk of code at once.

\hypertarget{the-environmenthistory-pane}{%
\subsubsection{The environment/history
pane}\label{the-environmenthistory-pane}}

The environment/history pane is, by default, located in the upper
right-hand panel of RStudio. The \textcolor{orange}{Environment} tab
shows any functions or data that you have in your
\textcolor{blue}{workspace}; that is, any objects created by running R
code. Up to now, your workspace only includes the function
\texttt{colFmt()}, which allows me to color text in Rmarkdown.

The \textcolor{orange}{History} tab shows all of the commands that have
been entered into the console.

While the environment/history panel can sometimes be helpful, it is
\textcolor{red}{not} a substitute for using R scripts or Rmarkdown!

\hypertarget{the-filesplotspackageshelp-pane}{%
\subsubsection{The files/plots/packages/help
pane}\label{the-filesplotspackageshelp-pane}}

The final panel, in the bottom right, by default hosts
files/plots/packages/help. This is where any plots that you generate
will be displayed, and any help files that you access will appear (more
on both of these later!).

\hypertarget{r-commands}{%
\subsection{R: commands}\label{r-commands}}

R commands are the lines of code you run; they relate
\textcolor{blue}{functions} to \textcolor{orange}{objects}. Functions
\textcolor{blue}{do} things, while objects \textcolor{orange}{store}
values. Everything in R is done by a function---for example, the code
you entered in before ran the \texttt{print()} function. Both of the
following commands return the same value:

\begin{Shaded}
\begin{Highlighting}[]
\DecValTok{45} \SpecialCharTok{+} \DecValTok{2}
\end{Highlighting}
\end{Shaded}

\begin{verbatim}
## [1] 47
\end{verbatim}

\begin{Shaded}
\begin{Highlighting}[]
\FunctionTok{print}\NormalTok{(}\DecValTok{45} \SpecialCharTok{+} \DecValTok{2}\NormalTok{)}
\end{Highlighting}
\end{Shaded}

\begin{verbatim}
## [1] 47
\end{verbatim}

\hypertarget{functions}{%
\subsubsection{Functions}\label{functions}}

Functions take in \textcolor{blue}{arguments}. This is how you, the
user, tell a function what to do. Some functions just perform an action,
and others return a \textcolor{orange}{value}. A function is accessed by
typing its name, followed by an open and closed set of parentheses: for
example, \texttt{read.csv()} is a function that can read in data in a
comma-separated file (.csv).

\textcolor{blue}{Arguments} to functions go between the parentheses, and
are separated by commas. You specify what values the arguments should
take on by using \texttt{=}; for example,
\texttt{read.csv(file\ =\ "births.csv",\ header\ =\ TRUE)} reads in the
data in \texttt{births.csv}, and treats the first line of the file as a
header (containing column names, not actual data).

Functions return \textcolor{orange}{values}, which are generally a
combination of objects, plots, and printouts. Printouts just appear on
your console, while objects and plots can be given names and stored for
later. The value of
\texttt{read.csv(file\ =\ "births.csv",\ header\ =\ TRUE)} is a special
R object called a \texttt{data.frame}, which is essentially equivalent
to a dataset.

\hypertarget{creating-objects}{%
\subsubsection{Creating objects}\label{creating-objects}}

R objects are (basically) the result of calling functions. However,
there are two general ways this can be done:

\begin{itemize}
\tightlist
\item
  \textcolor{blue}{loading data} (e.g., from a .txt or .csv file)
\item
  \textcolor{orange}{manipulating another object} using a function
\end{itemize}

You assign a value to an object using the special character
\texttt{\textless{}-}. The value, on the right-hand side, can now be
accessed using its object name (on the left-hand side).

I suggest naming your objects so that the name gives you some
information about the object---don't go too overboard, but this practice
has saved me lots of time. For example, \texttt{births} is a much more
informative name for a dataset than \texttt{x}.

\hypertarget{reading-in-a-dataset}{%
\subsubsection{Reading in a dataset}\label{reading-in-a-dataset}}

Our example for today is the King County births data, which you'll see
many times this quarter. First, you need to read the data into R. In
order to do this, you need to tell R where the data file lives on your
computer.

A \textcolor{blue}{working directory} is where R will read files from,
and also save files (like plots, if you choose). You might create a
folder on your computer for your homework assignments or your final
project, and store all relevant files there. So set your working
directory during an R session, use hte \texttt{setwd()} command. I'll
set my working directory right now to where the data is on my computer
--- yours will not be the same!

After we've told R where it can find the data, we can read the data
file.

\begin{Shaded}
\begin{Highlighting}[]
\DocumentationTok{\#\# load the FEV data}
\FunctionTok{setwd}\NormalTok{(}\StringTok{"C:/Users/ninag/Documents/GitHub/BIOST311\_2023/Datasets"}\NormalTok{)}
\NormalTok{births }\OtherTok{\textless{}{-}} \FunctionTok{read.csv}\NormalTok{(}\AttributeTok{file =} \StringTok{"births.csv"}\NormalTok{, }
                \AttributeTok{header =} \ConstantTok{TRUE}\NormalTok{)}
\DocumentationTok{\#\# check that you\textquotesingle{}ve read it in correctly:}
\FunctionTok{head}\NormalTok{(births)}
\end{Highlighting}
\end{Shaded}

\begin{verbatim}
##   sex plural age     race parity married  bwt smokeN drinkN firstep welfare
## 1   F      1  31 hispanic      4       1 3118      0      0       1       0
## 2   M      1  23    white      0       1 3466      0      0       1       0
## 3   F      1  24    black      1       0 3147      0      0       1       0
## 4   M      1  26 hispanic      2       1 3969      0      0       1       0
## 5   M      1  34    white      0       0 3005      0      0       1       0
## 6   M      1  15    black      0       0 2920      0      0       1       0
##   smoke drink wpre wgain education gestation
## 1     0     0  122    22         5        40
## 2     0     0  160    50        12        39
## 3     0     0  150    38        13        40
## 4     0     0  175    15        12        39
## 5     0     0  123    17        17        40
## 6     0     0  180    12         9        36
\end{verbatim}

What happened here? First, the \texttt{setwd()} command set my working
directory to the folder that holds all my datasets for this course.
Then, the \texttt{read.csv()} function read in the data; you then used
\texttt{\textless{}-} to assign those data, as a value, to the object
called \texttt{births}. Finally, you used the \texttt{head()} function
to print the first six rows of the \texttt{births} object.

\hypertarget{data-structures}{%
\subsection{Data structures}\label{data-structures}}

Reading a dataset into R isn't the end of the story---often, you'll need
to compute \textcolor{blue}{descriptive statistics},
\textcolor{orange}{plot} certain variables, and
\textcolor{cyan}{run statistical analyses}, sometimes on only part of
the data. Throughout this quarter, you'll learn tools to do all of these
things.

I mentioned earlier that vectors are the building block of R
objects---vectors are 1-dimensional, so they only have a length. You can
create a vector using the \texttt{c()} function:

\begin{Shaded}
\begin{Highlighting}[]
\NormalTok{my\_vector }\OtherTok{\textless{}{-}} \FunctionTok{c}\NormalTok{(}\DecValTok{1}\NormalTok{,}\DecValTok{2}\NormalTok{,}\DecValTok{3}\NormalTok{,}\DecValTok{4}\NormalTok{,}\DecValTok{5}\NormalTok{)}
\FunctionTok{print}\NormalTok{(my\_vector)}
\end{Highlighting}
\end{Shaded}

\begin{verbatim}
## [1] 1 2 3 4 5
\end{verbatim}

\textcolor{blue}{Data frames}, R's way of storing datasets, are
collections of vectors: each column of a data frame is a
\textcolor{orange}{variable}, which is stored as a vector; similarly,
each row of a data frame is an \textcolor{red}{observation}, which is
also stored as a vector.

You access values in vectors using open and closed square brackets
\texttt{{[}{]}}. For example, let's say I want to access the third
element in \texttt{my\_vector}. I can do this by telling R to pull out
the third element via \texttt{my\_vector{[}3{]}}. Be careful - if you
give an index that doesn't exist in the vector, you'll get \texttt{NA}.
(This is an example of when R lets you make a mistake without telling
you. Learning to watch out for these instances will become more
intuitive with practice.)

\begin{Shaded}
\begin{Highlighting}[]
\NormalTok{my\_vector[}\DecValTok{3}\NormalTok{] }\CommentTok{\# this is fine}
\end{Highlighting}
\end{Shaded}

\begin{verbatim}
## [1] 3
\end{verbatim}

\begin{Shaded}
\begin{Highlighting}[]
\NormalTok{my\_vector[}\DecValTok{6}\NormalTok{] }\CommentTok{\# this returns NA}
\end{Highlighting}
\end{Shaded}

\begin{verbatim}
## [1] NA
\end{verbatim}

You can access multiple values using the \texttt{c()} function.

\begin{Shaded}
\begin{Highlighting}[]
\NormalTok{my\_vector[}\FunctionTok{c}\NormalTok{(}\DecValTok{1}\NormalTok{,}\DecValTok{3}\NormalTok{)]}
\end{Highlighting}
\end{Shaded}

\begin{verbatim}
## [1] 1 3
\end{verbatim}

You access values in data frames using the same square
brackets---however, since data frames have rows and columns, you access
a value in the \texttt{i}th row and \texttt{j}th column using
\texttt{{[}i,\ j{]}}.

For example, access the 3rd element in the 5th column of the births
data:

\begin{Shaded}
\begin{Highlighting}[]
\NormalTok{births[}\DecValTok{3}\NormalTok{, }\DecValTok{5}\NormalTok{]}
\end{Highlighting}
\end{Shaded}

\begin{verbatim}
## [1] 1
\end{verbatim}

From your earlier call to the \texttt{head()} function, you'll notice
that this output corresponds to the 3rd observation on the variable
\texttt{parity}.

\hypertarget{r-packages}{%
\subsection{R packages}\label{r-packages}}

Since R is open source software, many people contribute to its
development. This comes in the form of \textcolor{blue}{packages}:
bundles of functions that supplement the functions always loaded into R.

You will use several packages extensively throughout this course. We'll
start with the \texttt{tidyverse} package, which is very useful for
manipulating data.

The first time you want to use a package, you'll have to install it.
Most packages can be installed using
\texttt{install.packages("package\ name")}.

Once the package is installed, each subsequent time you want to use
functions from the package, you have to load it using
\texttt{library}---and you have to do this \textcolor{red}{each time}
you close and re-open R or RStudio.

For example, load the \texttt{tidyverse} package:

\begin{Shaded}
\begin{Highlighting}[]
\CommentTok{\# run this next line (without the \#) if you haven\textquotesingle{}t installed the package yet}
\CommentTok{\# install.packages("tidyverse")}
\CommentTok{\# run this next line every time you open a new R session}
\FunctionTok{library}\NormalTok{(}\StringTok{"tidyverse"}\NormalTok{)}
\end{Highlighting}
\end{Shaded}

\begin{verbatim}
## -- Attaching core tidyverse packages ------------------------ tidyverse 2.0.0 --
## v dplyr     1.1.1     v readr     2.1.4
## v forcats   1.0.0     v stringr   1.5.0
## v ggplot2   3.4.1     v tibble    3.2.1
## v lubridate 1.9.2     v tidyr     1.3.0
## v purrr     1.0.1     
## -- Conflicts ------------------------------------------ tidyverse_conflicts() --
## x dplyr::filter() masks stats::filter()
## x dplyr::lag()    masks stats::lag()
## i Use the conflicted package (<http://conflicted.r-lib.org/>) to force all conflicts to become errors
\end{verbatim}

\hypertarget{data-wrangling}{%
\subsection{Data wrangling}\label{data-wrangling}}

Using the \texttt{tidyverse} package, you can pass the output of one
function directly to another --- that way, you don't have to save all
the intermediate objects as separate variables. This keeps things much
cleaner! To pass output from one function to the next, using the
\texttt{\%\textgreater{}\%} operator (referred to as ``pipe.'')

For example, suppose we want to take the \texttt{age} column of the
births dataset, and then look at some rows. We can pick columns using
\texttt{select()}. The \texttt{head()} function will show the first 6
rows, while we can choose arbitrary chunk of rows using
\texttt{slice()}. (Our input to \texttt{slice()} is \texttt{20:30},
which simply lists all integers from 20 to 30.) The
\texttt{\%\textgreater{}\%} operator links these commands together.

\begin{Shaded}
\begin{Highlighting}[]
\DocumentationTok{\#\# get the names of the data frame}
\FunctionTok{names}\NormalTok{(births)}
\end{Highlighting}
\end{Shaded}

\begin{verbatim}
##  [1] "sex"       "plural"    "age"       "race"      "parity"    "married"  
##  [7] "bwt"       "smokeN"    "drinkN"    "firstep"   "welfare"   "smoke"    
## [13] "drink"     "wpre"      "wgain"     "education" "gestation"
\end{verbatim}

\begin{Shaded}
\begin{Highlighting}[]
\DocumentationTok{\#\# look at just the age column}
\NormalTok{births }\SpecialCharTok{\%\textgreater{}\%} \FunctionTok{select}\NormalTok{(age) }\SpecialCharTok{\%\textgreater{}\%} \FunctionTok{head}\NormalTok{()}
\end{Highlighting}
\end{Shaded}

\begin{verbatim}
##   age
## 1  31
## 2  23
## 3  24
## 4  26
## 5  34
## 6  15
\end{verbatim}

\begin{Shaded}
\begin{Highlighting}[]
\DocumentationTok{\#\# access the 20th through 30th values of age}
\NormalTok{births }\SpecialCharTok{\%\textgreater{}\%} \FunctionTok{select}\NormalTok{(age) }\SpecialCharTok{\%\textgreater{}\%} \FunctionTok{slice}\NormalTok{(}\DecValTok{20}\SpecialCharTok{:}\DecValTok{30}\NormalTok{)}
\end{Highlighting}
\end{Shaded}

\begin{verbatim}
##    age
## 1   18
## 2   26
## 3   27
## 4   17
## 5   26
## 6   32
## 7   35
## 8   39
## 9   27
## 10  31
## 11  38
\end{verbatim}

The \texttt{names()} function returns the variable names in a data
frame.

The final major concept in manipulating datasets is
\textcolor{blue}{logical expressions}. The two logical values are
\texttt{TRUE} and \texttt{FALSE}; we can make comparisons that will
return these values. For example, is 3 \textgreater{} 5?

\begin{Shaded}
\begin{Highlighting}[]
\DecValTok{3} \SpecialCharTok{\textgreater{}} \DecValTok{5}
\end{Highlighting}
\end{Shaded}

\begin{verbatim}
## [1] FALSE
\end{verbatim}

You can now \textcolor{orange}{subset} your data by making logical
comparisons. For example, you might be interested only in those study
participants who are less than or equal to 20 years old. This means
you're interested in \textcolor{blue}{all columns} from the births data,
but only for the participants (i.e., \textcolor{red}{observations})
whose \texttt{age} value is \texttt{\textless{}=\ 20}. We perform
subsetting based on logical expressions using the \texttt{filter}
function.

\begin{Shaded}
\begin{Highlighting}[]
\DocumentationTok{\#\# subset the data, keep all columns}
\NormalTok{under20 }\OtherTok{\textless{}{-}}\NormalTok{ births }\SpecialCharTok{\%\textgreater{}\%} \FunctionTok{filter}\NormalTok{(age }\SpecialCharTok{\textless{}=} \DecValTok{20}\NormalTok{)}
\DocumentationTok{\#\# simple summary statistics}
\FunctionTok{summary}\NormalTok{(under20)}
\end{Highlighting}
\end{Shaded}

\begin{verbatim}
##      sex                plural       age           race          
##  Length:230         Min.   :1   Min.   :14.0   Length:230        
##  Class :character   1st Qu.:1   1st Qu.:18.0   Class :character  
##  Mode  :character   Median :1   Median :19.0   Mode  :character  
##                     Mean   :1   Mean   :18.7                     
##                     3rd Qu.:1   3rd Qu.:20.0                     
##                     Max.   :1   Max.   :20.0                     
##      parity          married            bwt           smokeN      
##  Min.   :0.0000   Min.   :0.0000   Min.   : 943   Min.   : 0.000  
##  1st Qu.:0.0000   1st Qu.:0.0000   1st Qu.:2829   1st Qu.: 0.000  
##  Median :0.0000   Median :0.0000   Median :3289   Median : 0.000  
##  Mean   :0.2217   Mean   :0.2826   Mean   :3216   Mean   : 1.148  
##  3rd Qu.:0.0000   3rd Qu.:1.0000   3rd Qu.:3628   3rd Qu.: 0.000  
##  Max.   :3.0000   Max.   :1.0000   Max.   :4593   Max.   :20.000  
##      drinkN           firstep          welfare            smoke       
##  Min.   :0.00000   Min.   :0.0000   Min.   :0.00000   Min.   :0.0000  
##  1st Qu.:0.00000   1st Qu.:0.0000   1st Qu.:0.00000   1st Qu.:0.0000  
##  Median :0.00000   Median :0.0000   Median :0.00000   Median :0.0000  
##  Mean   :0.01739   Mean   :0.4391   Mean   :0.01304   Mean   :0.1696  
##  3rd Qu.:0.00000   3rd Qu.:1.0000   3rd Qu.:0.00000   3rd Qu.:0.0000  
##  Max.   :2.00000   Max.   :1.0000   Max.   :1.00000   Max.   :1.0000  
##      drink               wpre           wgain          education    
##  Min.   :0.000000   Min.   : 75.0   Min.   :  0.00   Min.   : 0.00  
##  1st Qu.:0.000000   1st Qu.:115.2   1st Qu.: 26.00   1st Qu.:10.00  
##  Median :0.000000   Median :132.5   Median : 34.00   Median :11.50  
##  Mean   :0.008696   Mean   :140.4   Mean   : 35.59   Mean   :10.99  
##  3rd Qu.:0.000000   3rd Qu.:160.0   3rd Qu.: 43.75   3rd Qu.:12.00  
##  Max.   :1.000000   Max.   :290.0   Max.   :149.00   Max.   :15.00  
##    gestation    
##  Min.   :26.00  
##  1st Qu.:37.00  
##  Median :39.00  
##  Mean   :38.63  
##  3rd Qu.:40.00  
##  Max.   :45.00
\end{verbatim}

The \texttt{summary()} function you just used displays simple
descriptive statistics. This is a great way to get a quick snapshot of
the data, and make sure they have been read in correctly. Look at the
descriptive statistics for \texttt{wgain}. Do you notice anything
interest about this variable? What are its largest and smallest values?

REPLACE WITH YOUR ANSWER TO THIS QUESTION

\hypertarget{creating-a-new-variable}{%
\subsubsection{Creating a new variable}\label{creating-a-new-variable}}

You can also use the \texttt{tidyverse} package to create new variables
using the \texttt{mutate()} function. A baby is considered preterm if it
is born at 36 weeks gestation or earlier. We'll create a new binary
variable called \texttt{preterm} that is 1 for preterm births and 0
otherwise. We need to use the \texttt{\textless{}-} symbol to save our
altered dataset to the variable called \texttt{births}.

\begin{Shaded}
\begin{Highlighting}[]
\NormalTok{births }\OtherTok{\textless{}{-}}\NormalTok{ births }\SpecialCharTok{\%\textgreater{}\%} \FunctionTok{mutate}\NormalTok{(}\AttributeTok{preterm =} \FunctionTok{ifelse}\NormalTok{(gestation }\SpecialCharTok{\textless{}=} \DecValTok{36}\NormalTok{, }\DecValTok{1}\NormalTok{, }\DecValTok{0}\NormalTok{))}
\end{Highlighting}
\end{Shaded}

We used \texttt{ifelse()} to create the new binary variable: this takes
a logical comparison (e.g., is \texttt{gestation} less than or equal to
36?), what to assign if the expression is true (e.g., \texttt{1}), and
what to assign if the expression is false (e.g., \texttt{0}).

\emph{This does not, however, save these variables into the underlying
.csv file: for that, you'll need a different function,
\texttt{write.csv()}}.

\hypertarget{getting-help}{%
\subsection{Getting help}\label{getting-help}}

Until now, you've had functions provided to you, with the relevant
arguments filled out. But what if you don't know what arguments a
function has?

Typing \texttt{?} in the console, followed by the function name, calls
up the help file for the function. All R functions are required to have
help files.

\begin{Shaded}
\begin{Highlighting}[]
\NormalTok{?summary}
\end{Highlighting}
\end{Shaded}

\begin{verbatim}
## starting httpd help server ... done
\end{verbatim}

If you don't know the name of the function, you aren't out of luck. If
you know what you'd like to do, you can use \texttt{??}, followed by a
one-word description, to pull up all help files that mention the word.

\begin{Shaded}
\begin{Highlighting}[]
\NormalTok{??summarize}
\end{Highlighting}
\end{Shaded}

Finally, Google, Stack Exchange, Canvas, and your friends are your best
resource: if you have an R problem, I can almost guarantee that someone
else has had it before, and the answer is either on the internet or with
your friend!

\hypertarget{conclusion}{%
\subsection{Conclusion}\label{conclusion}}

You're now on your journey to becoming true data scientists!

Today, you've learned that R is a powerful software package that gives
you flexible options for analyzing your data. You've also learned how to
do simple arithmetic, read in a dataset, create objects, add variables
to data sets, calculate simple descriptive statistics, access help
files, and load packages.

While it may be frustrating at times, learning R is worth your time.
Additionally, using RMarkdown and RStudio is a great first step to
making sure that all of your analyses are reproducible---and being fully
reproducible is important for science!

\end{document}
