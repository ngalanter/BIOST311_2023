% Create a Table of Contents in Beamer
\documentclass[10pt,t]{beamer}
% Theme choice:
\usetheme{Singapore}
\useoutertheme{sidebar}
\usecolortheme{seahorse}
\setbeamercolor{titlelike}{bg=white}
\setbeamercolor{frametitle}{bg=white}
%\setbeamertemplate{frametitle}[default][left]
\setbeamertemplate{navigation symbols}{}
\setbeamertemplate{footline}{\begin{flushright}\small \insertframenumber\end{flushright}}


\usepackage{graphicx}
\usepackage{amsmath}
\usepackage{amsfonts}
\usepackage{amssymb}
\usepackage{amsthm}
\usepackage{ulem}
\usepackage{listings}

% Title page details: 
\title{Quiz 2 Comments} 
\author{Nina Galanter}
\date{\today}


\begin{document}
	% Title page frame
	\begin{frame}
	\titlepage 
\end{frame}

\begin{frame}{Question 4}
	Which of the following measures of spread is most affected by outliers (data points that differ significantly from other observations)?
	\medskip
	\begin{enumerate}
		\item Interquartile range
		\medskip
		\item Standard Deviation
		\medskip
		\item Variance
		\medskip
		\item {\color{green}Range}
	\end{enumerate}
\medskip
\begin{itemize}
	\item Interquartile range is the only measure that's robust to outliers
	\smallskip
	\item Standard Deviation is the square root of variance, so both are essentially equally affected by outliers
	\smallskip
	\item Range is more affected than variance/standard deviation
\end{itemize}
\end{frame}

\begin{frame}{Question 4}
	
	\vspace{-5 mm}
	Here we have 100 observations between 0 and 1, and 1 observation at 10. Without the outlier, the range is {\color{blue}0.999} and the standard deviation is {\color{violet}0.307}. With the outlier, the range is {\color{blue}10.000} and the standard deviation is {\color{violet}0.992}. 

\begin{figure}
	\centering
	\includegraphics[scale = 0.45]{range_vs_sd}
\end{figure}

\end{frame}

\begin{frame}{Question 6}
	I am interested in the association between high blood pressure and coronary artery disease. I have access to the University of Washington Medical Center electronic health record database. I randomly sample 10,000 adult patients from the database, and record (1) whether or not they have high blood pressure and (2) whether or not they have ever had a cardiovascular event. 
	\medskip
	
	This could best be characterized as which of the following study designs? 
	\medskip
	\begin{enumerate}
		\item Prospective cohort study
		\medskip
		\item Cross-sectional study
		\medskip
		\item {\color{green}Retrospective cohort study}
		\medskip
		\item Case-control study
	\end{enumerate}


\end{frame}

\begin{frame}{Question 6}
	\vspace{-5 mm}
	
	Better wording:
	\medskip
	
	I am interested in the association between high blood pressure and coronary artery disease. I have access to the University of Washington Medical Center electronic health record database. I randomly sample 10,000 adult patients from the database, and record (1) whether or not they have high blood pressure and (2) whether or not they have ever had a cardiovascular event \textbf{afterwards}. 
	\medskip
	

	\begin{itemize}
		\item In an electronic health record database, records generally consist of patient demographics and "events"
		\medskip
		\item Events can be appointments, surgeries, emergency room visits, prescriptions, tests, phone calls, etc
		\medskip
		\item We have information across many time-points for most people
		\medskip
		\item So it's possible to follow people in time from the exposure to see if they had the outcome
	\end{itemize}
	
\end{frame}

\begin{frame}{Question 9}
	Cohort studies are more efficient than case-control studies for studying rare outcomes. 
	
	\smallskip
	\begin{enumerate}
		\item True
		\item {\color{green}False}
	\end{enumerate}
	
	\medskip
	
	\begin{itemize}
		\item For cohort studies, we sample only people \textbf{without} the outcome
		\medskip
		\begin{itemize}
			\item The number of people with the outcome will be however many eventually develop it
			\smallskip
			\item If the outcome is rare, we need many people in our initial sample to make sure that enough people eventually get the outcome
		\end{itemize}
	\medskip
		\item  For case-control studies, we sample some people \textbf{who've had the outcome} and some who haven't
		\medskip
			\begin{itemize}
			\item We control (to the extent we are able to recruit people) how many people have the outcome
			\smallskip
			\item Our sample can be pretty small to have enough people with the outcome
		\end{itemize}
	\end{itemize}
\end{frame}

\end{document}
